\documentclass{warpdoc}
\newlength\lengthfigure                  % declare a figure width unit
\setlength\lengthfigure{0.158\textwidth} % make the figure width unit scale with the textwidth
\usepackage{psfrag}         % use it to substitute a string in a eps figure
\usepackage{subfigure}
\usepackage{rotating}
\usepackage{pstricks}
\usepackage[innercaption]{sidecap} % the cute space-saving side captions
\usepackage{scalefnt}
\usepackage{bm}
\usepackage{amsmath}

%%%%%%%%%%%%%=--NEW COMMANDS BEGINS--=%%%%%%%%%%%%%%%%%%%%%%%%%%%%%%%%%%
\newcommand\frameeqn[1]{\fbox{$\displaystyle #1$}}
\newcommand{\alb}{\vspace{0.1cm}\\} % array line break
\newcommand{\bigfrac}{\displaystyle\frac}
\newcommand{\mfd}{\displaystyle}
\renewcommand{\fontsizefigure}{\scalefont{0.95}}
\renewcommand{\fontsizetable}{\scalefont{0.85}}
\newcommand{\ns}{{n_{\rm s}}}
\newcommand{\nion}{{n_{\rm i}}}
\newcommand{\nns}{{n_{\rm ns}}}
\newcommand{\ncs}{{n_{\rm cs}}}
\newcommand{\ev}{{e_{\rm v}}}
\newcommand{\evzero}{{e_{\rm v}^0}}
\newcommand{\kappaev}{\kappa_{{\rm v}}}
\newcommand{\visc}{\eta}
\newcommand{\nd}{3}
\newcommand{\tauvt}{\tau_{\rm vt}}
\renewcommand{\vec}[1]{\bm{#1}}
\setcounter{tocdepth}{3}
\let\citen\cite



%%%%%%%%%%%%%=--NEW COMMANDS ENDS--=%%%%%%%%%%%%%%%%%%%%%%%%%%%%%%%%%%%%



\author{
  Bernard Parent
}

\email{
  bernparent@gmail.com
}

\department{
  Dept. Aerospace and Mechanical Engineering	
}

\institution{
  University of Arizona
}

\title{
  Charged Species Transport Equations in a Weakly-Ionized Gas  
}

\date{
  2014--2016
}

%\setlength\nomenclaturelabelwidth{0.13\hsize}  % optional, default is 0.03\hsize
%\setlength\nomenclaturecolumnsep{0.09\hsize}  % optional, default is 0.06\hsize

\nomenclature{
  \begin{nomenclaturelist}{Roman symbols}
   \item[$a$] speed of sound of the neutrals
   \item[$\vec{B}$] magnetic field vector
   \item[$C_k$] particule charge of $k$th species
   \item[$\vec{E}$] electric field vector
   \item[$\vec{E}^k$] the electric field in the $k$th species reference frame, $\vec{E}+\vec{V}^k\times\vec{B}$
   \item[$E^\star$] reduced electric field in electron reference frame, $|\vec{E}+\vec{V}^{\rm e}\times \vec{B}|/N$
   \item[$e$] elementary charge
   \item[$\vec{G}$]  vector of diffusion variables
   \item[$h_{k}$] specific enthalpy of $k$th species at its temperature $T_k$ excluding nitrogen vibrational energy and heat of formation
   \item[$h_{k}^0$] heat of formation of $k$th species 
   \item[$h_{\rm n}$] specific enthalpy of the neutrals at the temperature $T$ excluding nitrogen vibrational energy but including the heat of formation
   \item[$\vec{H}$] vector including extra convection terms needed for the charged species 
   \item[$\vec{J}$] current density vector
   \item[$k$] turbulence kinetic energy
   \item[$k_{\rm B}$] Boltzmann constant
   \item[$K$] diffusion matrix
   \item[$M_k$] mass of $k$th species within volume $v$ 
   \item[$m_k$] particule mass of $k$th species 
   \item[$\ns$] number of species (including charged species)
   \item[$\nns$] number of neutral species 
   \item[$\ncs$] number of charged species 
   \item[$N_k$] species number density 
   \item[$N$] total number density of the mixture, $\sum_k N_k$
   \item[$P_k$] species partial pressure, $\rho_k R_k T_k$
   \item[$P_{\rm n}$] pressure of all neutrals
   \item[$P^\star$] effective pressure including turbulence and electron energy contributions, $\sum_k P_k + \frac{2}{3} \rho k$
   \item[$\rm Pr$] adjusted Prandtl number, $\frac{\visc}{\kappa}\left( c_p + w_{\rm N_2} \frac{\partial \evzero}{\partial T}\right)$
   \item[$\rm Pr_t$] turbulent Prandtl number, typically set to 0.9
   \item[$Q_{\rm b}$] electron beam power deposited
   \item[$Q_{k}$] turbulence kinetic energy production term
   \item[$Q_{\omega}$] source term of the $\omega$ transport equation, $\frac{5}{9}  Q_k - \frac{5}{6} \rho \epsilon$
   \item[$Q_{\rm J}^{\rm e}$] electron Joule heating, $\frac{1}{\mu_{\rm e}}|C_{\rm e}|N_{\rm e}|\vec{V}^{\rm e}-\vec{V}^{\rm n}|^2$
   \item [$q_i^{\rm v}$] heat flux associated with the nitrogen vibrational energy mode along $i$th dimension
   \item [$q_i^k$] heat flux associated with the $k$th species along $i$th dimension
   \item [$q_i$] heat flux associated with the neutral species along $i$th dimension
   \item[$R_k$] gas constant of $k$th species
   \item[$s_k$] sign of the charge of species $k$ (either $+1$ for the positive species or $-1$ for the negative species)
   \item[$\vec{S}$] source term vector
   \item[$\rm Sc_t$] turbulent Schmidt number, typically set to 1
   \item[$T$] neutrals and ions temperature
   \item[$T_k$] temperature of $k$th species
   \item[$T_{\rm v}$] nitrogen vibrational temperature
   \item[$T_{\rm e}$] electron temperature
   \item[$t$] time
   \item[$\vec{U}$] vector of conserved variables
   \item[$v$] volume (used in first law of thermo)
   \item[$\vec{V}^{\rm n}$] velocity of the neutrals   
   \item[$\vec{V}^r$] the $r$th species velocity   
   \item[$W_k$] mass production of species $k$ per unit volume due to chemical reactions
   \item[$w_k$] mass fraction of $k$th species, $\rho_k/\rho$
   \item[$x_i$] Cartesian coordinates  
   \item[$x,y,z$] Cartesian coordinates  
   \item[$X_i$] grid index along the $i$th dimension  
   \item[$X_{i,j}$] $\partial X_i/\partial x_j$  
   \item[$\vec{Y}$] vector function of electric field and current density  
   \item[$Z$] matrix related to the unsteady terms
   \item[$z$] row number of the $\rm N_2$ mass conservation equation
  \end{nomenclaturelist}


  \begin{nomenclaturelist}{Greek symbols}
   \item[$\alpha$] non-dimensional mass-based ambipolar tensor 
   \item[$\alpha^\prime$] non-dimensional ambipolar tensor 
   \item[$\beta_k^+$] equal to 1 if species $k$ is a positive ion, 0 otherwise 
   \item[$\beta_k^-$] equal to 1 if species $k$ is a negative ion or electron, 0 otherwise 
   \item[$\beta_k^{\rm e}$] equal to 1 if species $k$ is an electron, 0 otherwise 
   \item[$\beta_k^{\rm n}$] equal to 1 if species $k$ is a neutral, 0 otherwise 
   \item[$\beta_k^{\rm c}$] equal to 1 if species $k$ is a charged species, 0 otherwise 
   \item[$\gamma$] secondary emission coefficient
   \item[$\delta_{rk}$]        Kronecker delta
   \item[$\vec{\Delta V}^{r}$] velocity contribution due to magnetic field for the $r$th species    
   \item[$\epsilon$] dissipation rate of the turbulence kinetic energy, $k\omega$
   \item[$\epsilon_0$]   permittivity of free space
   \item[$\zeta_{\rm v}$]  fraction of the electron Joule heating that is consumed in the excitation of the vibration levels of the nitrogen molecule
   \item[$\eta$]       viscosity of the neutrals mixture, obtained from Wilke's mixing rule
   \item[$\eta^\star$]       effective viscosity including turbulence contribution, $\eta+\eta_{\rm t}$
   \item[$\eta_{\rm t}$]       turbulence viscosity, $0.09 \rho k / \omega$
   \item[$\eta^\star_k$]     $k$ transport equation diffusion coefficient, $\eta+\frac{1}{2} \eta_{\rm t}$    
   \item[$\eta^\star_\omega$]       $\omega$ transport equation diffusion coefficient, $\eta+\frac{1}{2} \eta_{\rm t}$
   \item[$\Theta_{\rm v}$] nitrogen characteristic vibration temperature, 3353~K
   \item[$\kappa$]       thermal conductivity of the neutrals mixture obtained from the Mason and Saxena relation
   \item[$\kappa^\star$]       effective thermal conductivity including turbulence contribution, $c_p \left( \frac{\visc}{\rm Pr} +\frac{\visc_{\rm t}}{{\rm Pr_t}}\right)$
   \item[$\kappa_{\rm e}$]       electron thermal conductivity, $\frac{5}{2} k_{\rm B}^2 N_{\rm e}\mu_{\rm e} T_{\rm e}/|C_{\rm e}|$
   \item[$\kappa_{\rm v}$]       nitrogen vibrational thermal conductivity, $w_{\rm N_2} \frac{\eta}{\rm Pr} \frac{\partial e_{\rm v}}{\partial T_{\rm v}}$
   \item[$\kappa_{\rm v}^\star$]       effective nitrogen vibrational thermal conductivity including turbulence contribution, $w_{\rm N_2} \left( \frac{\visc}{\rm Pr}+\frac{\visc_{\rm t}}{\rm Pr_t}  \right)\frac{\partial \ev}{\partial T_{\rm v}}$
   \item[$\mu_k$]        mobility of the $k$th species
   \item[$\wtilde{\mu}^k$] tensor mobility of the $k$th species
   \item[$\nu_k$]        mass diffusion coefficient for the neutral species, determined from Wilke's rule
   \item[$\nu_k^\star$]        effective mass diffusion coefficient for the neutral species including turbulence contribution, $\nu_k + \frac{\eta_{\rm t}}{\rm Sc_t}$
   \item[$\rho$]       density of the mixture, $\sum_k \rho_k$
   \item[$\rho_{\rm c}$]       net charge density, $\sum_k C_k N_k$
   \item[$\rho_{\rm c}$]       electron partial mass density
   \item[$\rho_{\rm n}$]       density of all neutrals
   \item[$\rho_k$]       partial mass density of $k$th species
   \item[$\sigma$]       conductivity, $\sum_k |C_k| N_k \mu_k$
   \item[$\tau_{\rm vt}$]  nitrogen vibration-translation relaxation time
   \item[$\tau_{ji}$]  viscous stress in the direction of the $x_i$ coordinate and acting on surface perpendicular to $x_j$  
   \item[$\chi$]       coordinate perpendicular to the surface and pointing towards the fluid
   \item[$\psi$]  non-dimensional tensor related to the Navier-Stokes stresses
   \item[$\Omega$] inverse of the metrics Jacobian
   \item[$\omega$] specific dissipation rate of the turbulence kinetic energy
  \end{nomenclaturelist}

}


\abstract{
abstract
}

\begin{document}
  \pagestyle{headings}
  \pagenumbering{arabic}
  \setcounter{page}{1}
%%  \maketitle
  \makewarpdoctitle
%  \makeabstract
  \tableofcontents
%%  \makenomenclature
%%  \listoftables
%%  \listoffigures


\section{Introduction}

This document is based on the approach outlined in Refs.\ \cite{jcp:2014:parent,jcp:2015:parent} in which a new set of electron and ion transport equations is proposed that is advantaged over the conventional equations by being free of stiffness. This permits the use of large integration steplengths to be used and in as much as a thousandfold increase in computational efficiency.


\section{Physical Model}

Let us first outline the physical model from which the recast computationally-efficient set of transport equations will be subsequently derived. Commonly referred to as the ``fluid model'' or ``drift-diffusion model'', the physical model under consideration treats the neutrals and each charged species as independent fluids with their own velocities interacting with the other fluids through collision forces. In the presence of a magnetic field, the drift-diffusion model yields the following mass conservation equation for each charged species (either electrons, positive ions, or negative ions):
%
\begin{equation}
  \frac{\partial N_k}{\partial t} + \sum_{i=1}^3 \frac{\partial}{\partial x_i} N_k \vec{V}_i^k = W_k
  \label{eqn:massconservation}
\end{equation}
%
where $k$ is an index associated with the species to be solved and where the species velocity $\vec{V}^k$ is obtained from:
%
\begin{equation}
  \vec{V}^k=\vec{V}^{\rm n} + s_k \mu_k \left(\vec{E}+\vec{V}^k \times \vec{B}\right)-\frac{\mu_k}{|C_k| N_k}\nabla P_k
 \label{eqn:Vvector}
\end{equation}
% 
The latter expression for the species velocity can be obtained from the momentum equation by assuming that the terms related to inertia change and to collision forces between charged species are negligible compared to the terms related to the collision forces between the charged species and the neutrals. 

In the mass and momentum equations above, $N_k$ is the number density of species $k$, $\vec{V}_i^k$ is the $i$th component of the $k$th species velocity including drift and diffusion, $\vec{V}^{\rm n}$ is the neutrals velocity vector including drift and diffusion, $W_k$ is the source term containing all chemical reactions, and $P_k$ is the partial pressure. As well, $s_k$ is the species sign  (equal to $+1$ for the positive ions and to $-1$ for the electrons and the negative ions), $C_k$ is the species charge  (equal to $-e$ for the electrons, to $+e$ for the singly-charged positive ions, to $-2e$ for the doubly-charged negative ions, etc, with $e$ the elementary charge), $\vec{E}$ is the electric field vector, $\vec{B}$ the magnetic field vector, and $\mu_k$ the species mobility.
 
It can be convenient to rewrite the  charged species velocity vector in tensor form as follows:
%
\begin{equation}
  \vec{V}^{k}_i = \vec{V}^{\rm n}_i+\sum_{j=1}^\nd s_k \wtilde{\mu}^k_{ij}  \vec{E}_j^{\rm n}
             - \sum_{j=1}^\nd  \frac{\wtilde{\mu}^{k}_{ij}}{|C_k| N_k} \frac{\partial P_k}{\partial x_j}
  \label{eqn:V}
\end{equation}
%
with $\vec{E}^{\rm n}$ being the effective electric field in the neutrals reference frame:
%
\begin{equation}
\vec{E}^{\rm n} \equiv \vec{E}+\vec{V}^{\rm n} \times \vec{B}
\label{eqn:En}
\end{equation}
%
and with the mobility tensor equal to:
%
\begin{equation}
\!\!\!
\begin{array}{l}\mfd
\wtilde{\mu}^k  =\frac{\mu_k}{1+\mu_k^2|\vec{B}|^2}\left[\!\!\begin{array}{ccc} 
      1+\mu_k^2 \vec{B}_1^2 
     & \mu_k^2\vec{B}_1\vec{B}_2+s_k \mu_k \vec{B}_3  
     & \mu_k^2\vec{B}_1\vec{B}_3-s_k \mu_k \vec{B}_2 \alb
      \mu_k^2\vec{B}_1\vec{B}_2-s_k\mu_k\vec{B}_3 & 1+\mu_k^2\vec{B}_2^2 &  \mu_k^2\vec{B}_2\vec{B}_3+s_k\mu_k\vec{B}_1  \alb
      \mu_k^2\vec{B}_1\vec{B}_3 +s_k\mu_k\vec{B}_2 & \mu_k^2 \vec{B}_2\vec{B}_3-s_k\mu_k\vec{B}_1  & 1+\mu_k^2\vec{B}_3^2 
    \end{array} \!\!\!\!\right]
\end{array}
\label{eqn:mutilde}
\end{equation}
%
In the latter, the magnetic field $\vec{B}$ corresponds to the externally applied magnetic field, as the induced magnetic field can be shown to have negligible impact on many weakly-ionized plasmas (the so-called low magnetic Reynolds number approximation). When the induced magnetic field plays a negligible role, and when the applied (external) magnetic field does not vary in time, it can be demonstrated that the Maxwell equations reduce to the solution of Gauss's law:
%
\begin{equation}
\sum_{j=1}^3 \frac{\partial }{\partial x_j} \epsilon_r \vec{E}_j=\frac{1}{\epsilon_0} \sum_{k=1}^\ns C_k N_k 
\label{eqn:gauss}
\end{equation}
%
in which the electric field vector can be expressed in terms of a potential function as follows:  
%
\begin{equation}
  \vec{E}_j=-\frac{\partial \phi}{\partial x_j}
  \label{eqn:potential}
\end{equation}
%
The electric field potential $\phi$ exists as long as the curl of the electric field is zero, which is the case when the magnetic field does not vary in time.  
Although not required to solve the system of equations outlined above, one physical parameter that is often used when analyzing plasma flowfields is the current density $\vec{J}$, which is defined as:
%
\begin{equation}
  \vec{J}_i \equiv \sum_{k=1}^\ns C_k N_k  \vec{V}^k_i
 \label{eqn:Jdefinition}
\end{equation}
%
After substituting in the latter the velocity tensor from Eq.\ (\ref{eqn:V}), the following expression for the current can be obtained:
%
\begin{equation}
\begin{array}{l}
\mfd  \vec{J}_i = \sum_{j=1}^3  \wtilde{\sigma}_{ij} \vec{E}_j^{\rm n} 
             - \sum_{j=1}^3 \sum_{k=1}^\ns s_k \wtilde{\mu}^k_{ij}  \frac{\partial P_k}{\partial x_j}
+ \rho_{\rm c} \vec{V}_i^{\rm n} 
\end{array}
\label{eqn:J}
\end{equation}
%
in which the tensor conductivity and the net charge density are defined as:
%
\begin{equation}
\begin{array}{l}
\mfd
\wtilde{\sigma}\equiv\sum_{k=1}^\ns |C_k| N_k \wtilde{\mu}^k
\end{array}
\label{eqn:sigmatilde}
\end{equation}
%
%
\begin{equation}
\rho_{\rm c}\equiv\sum_{k=1}^\ns C_k N_k
\label{eqn:rhoe}
\end{equation}
%
Finally, a constitutive relation that is needed to close the system of equations is the ideal gas law which yields the partial pressure given the number density and the temperature:
%
\begin{equation}
P_k = N_k k_{\rm B} T_k
\label{eqn:Pk}
\end{equation}
%
where $T_k$ is the species temperature.

Commonly used to simulate weakly-ionized plasmas in the presence of magnetic field, the physical model outlined above can predict accurately not only quasi-neutral phenomena such as ambipolar diffusion and ambipolar drift but also non-neutral phenomena within cathode and anode sheaths. It can also be used to simulate multicomponent plasmas in which there are several types of ions (either negative or positive) as well as unsteady plasmas in which the displacement current is significant. Nonetheless, it is pointed out that the physical model used herein is subject to several assumptions, with the most critical being the following: (i) the forces due to collisions between charged species are small compared to forces due to collisions  between the charged species and the neutrals; (ii) the induced magnetic field is negligible; (iii) within the momentum equation, the terms related to the inertia change are negligible compared to the terms related to collision forces. As was demonstrated in Ref.\ \cite{jcp:2011:parent}, such assumptions are well justified as long as the plasma remains weakly-ionized (i.e. the ionization fraction should remain lower than $10^{-3}$ or so), which is the case for a wide variety of plasmas used in industrial applications.     



\section{Conventional Governing Equations}

When using digital computers, the conventional approach to simulate the drift-diffusion physical model outlined in the previous section  consists of solving a transport equation for each charged species along with the potential equation obtained from Gauss's law (see for instance Refs.\ \cite{jcp:2004:surzhikov,jpp:2008:poggie,jap:2009:shang,book:2012:surzhikov,aiaaconf:2014:surzhikov}). The charged species transport equation can be derived from the mass conservation equations as outlined in Eq.\ (\ref{eqn:massconservation}) with the species velocity from the momentum equation applicable to a weakly-ionized plasma as shown in Eq.\ (\ref{eqn:V}). This  yields the following:
%
\begin{equation}
  \frac{\partial N_k}{\partial t} + \sum_{i=1}^3 \frac{\partial}{\partial x_i} N_k\left(\vec{V}^{\rm n}_i+\sum_{j=1}^\nd s_k \wtilde{\mu}^k_{ij}  \vec{E}_j^{\rm n}
             - \sum_{j=1}^\nd  \frac{\wtilde{\mu}^{k}_{ij}}{|C_k| N_k} \frac{\partial P_k}{\partial x_j}\right) = W_k
\end{equation}
%
When the magnetic field is strong (resulting in an electron Hall parameter $\mu_{\rm e}|\vec{B}|$ approaching or exceeding 1), and when the transport equations are solved through an implicit integration strategy, it is beneficial to the stability of the method to extract from the pressure gradient terms the diffusion terms that are diagonally dominant. This can be achieved by first subtracting and adding a pressure gradient term on the LHS as follows:
%
\begin{align} 
 \frac{\partial N_k}{\partial t} &+ \sum_{i=1}^3 \frac{\partial}{\partial x_i} N_k\left(\vec{V}^{\rm n}_i+\sum_{j=1}^\nd s_k \wtilde{\mu}^k_{ij}  \vec{E}_j^{\rm n}
             - \sum_{j=1}^\nd  \frac{\wtilde{\mu}^{k}_{ij}}{|C_k| N_k} \frac{\partial P_k}{\partial x_j}
-\frac{\mu_k}{|C_k| N_k} \frac{\partial P_k}{\partial x_i} +\frac{\mu_k}{|C_k| N_k} \frac{\partial P_k}{\partial x_i}
\right)\nonumber\\ &= W_k
\end{align}
%
Then, expand the partial pressure using the ideal gas law $P_k=N_k k_{\rm B} T_k$ and reformat:
%
\begin{align}
  \frac{\partial N_k}{\partial t} &+ \sum_{i=1}^3 \frac{\partial}{\partial x_i} N_k\left(\vec{V}^{\rm n}_i+\sum_{j=1}^\nd s_k \wtilde{\mu}^k_{ij}  \vec{E}_j^{\rm n}
             - \sum_{j=1}^\nd  \frac{\wtilde{\mu}^{k}_{ij}-\delta_{ij} \mu_k}{|C_k| N_k} \frac{\partial P_k}{\partial x_j}\right)\nonumber\\
&-\sum_{i=1}^3 \frac{\partial}{\partial x_i} \left(\frac{T_k k_{\rm B} \mu_k}{|C_k| } \frac{\partial N_k}{\partial x_i} \right)
=W_k+
\sum_{i=1}^3 \frac{\partial}{\partial x_i} \left(\frac{N_k k_{\rm B} \mu_k}{|C_k| } \frac{\partial T_k}{\partial x_i}
\right)  
\end{align}
%
with $\delta_{ij}$ the Kronecker delta which is equal to 1 should $i=j$ and to 0 otherwise. We can write the latter in matrix form as follows:
%
\begin{equation}
 R = Z \frac{\partial U}{\partial t} + \sum_{i=1}^3 \frac{\partial}{\partial x_i} A_i U  - \sum_{i=1}^3  \frac{\partial}{\partial x_i}  \left( K \frac{\partial U}{\partial x_i}\right)  -S
\end{equation}
%
where $R$ is the residual vector and the other matrices correspond to:
%
\begin{align}
 \left[U\right]_k&=N_k \alb
 \left[A_i\right]_{k,k} &= \vec{V}^{\rm n}_i+\sum_{j=1}^\nd s_k \wtilde{\mu}^k_{ij}  \vec{E}_j^{\rm n}
             - \sum_{j=1}^\nd  \frac{\wtilde{\mu}^{k}_{ij}-\delta_{ij} \mu_k}{|C_k| N_k} \frac{\partial P_k}{\partial x_j}  \alb
 \left[K\right]_{k,k} &= \frac{T_k k_{\rm B} \mu_k}{|C_k|} \alb
 \left[Z\right]_{k,k} &= 1 \alb
 \left[S\right]_k &=W_k+
\sum_{i=1}^3 \frac{\partial}{\partial x_i} \left(\frac{N_k k_{\rm B} \mu_k}{|C_k| } \frac{\partial T_k}{\partial x_i}
\right) 
\end{align}
%
where the notation $[M]_{k,k}$ denotes the diagonal element on the $k$th row of the matrix $M$ while the notation $[F]_k$ refers to the element on the $k$th row of the vector $F$. The latter yields a diagonally-dominant diffusion matrix $K$ even at high electron Hall parameter, which is a necessary condition for stable integration using an implicit method. As well, extracting the non-diagonal diffusion terms from matrix $K$ and inserting them in the convection matrix $A$ permits standard central stencils to be used when discretizing the diffusion terms. However, should they include cross-derivatives, the diffusion terms would require non-standard upwinded stencils or they would lead to spurious oscillations at a high Hall parameter (see Ref.\ \cite{jcp:2011:parent} for more details on this point).   

The electric field is obtained from the potential as in Eq.\ (\ref{eqn:potential}) which itself is found by integrating the potential equation concurrently to the mass conservation equations. To ensure that Gauss's law is satisfied within the non-neutral regions, it is necessary to obtain the potential equation from Gauss's law by substituting Eq.\ (\ref{eqn:potential}) into Eq.\ (\ref{eqn:gauss}):
%
\begin{equation}
\sum_{j=1}^3 \frac{\partial }{\partial x_j}\left(\epsilon_r \frac{\partial \phi}{\partial x_j}\right)=-\frac{\rho_{\rm c}}{\epsilon_0}  
\label{eqn:potentialgauss}
\end{equation}
%  
The latter constitute what is here denoted as the ``conventional governing equations'', which are commonly used to solve weakly-ionized plasmas (either quasi-neutral or non-neutral) in the presence of a magnetic field using discrete methods.










\section{Addition of Gauss Law Terms}


The ``conventional governing equations'' outlined in the previous section are well known to be exceptionally stiff. Such stiffness has been observed to be independent of the type of integration strategy used, either explicit or fully-implicit. As first outlined in Ref.\ \cite{jcp:2013:parent}, the stiffness originates from the potential equation based on Gauss's law being particularly sensitive to small errors in the electron or ion densities whenever the plasma becomes quasi-neutral. One approach that has been shown successful in relieving the stiffness is by rewriting the governing equations such that the electric field is obtained from a potential based on Ohm's law rather than Gauss's law (see Refs.\ \cite{jcp:2013:parent} and \cite{jcp:2014:parent}). As well, to ensure that Gauss's law is satisfied some source terms need to be added to the positive ion transport equations. 

We here generalize the approach proposed in Ref.\ \cite{jcp:2014:parent} to a plasma in magnetic field. To do so, it is convenient to first define $\Delta \vec{V}^k$ as the difference between the velocity of species $k$ and the velocity of species $k$ should the magnetic field be zero:
%
\begin{equation}
\Delta \vec{V}^k_i \equiv \vec{V}^k_i - \left(\vec{V}^{\rm n}_i+ s_k \mu_k  \vec{E}_i
             -   \frac{\mu_k}{|C_k| N_k} \frac{\partial P_k}{\partial x_i}\right)
\label{eqn:deltaV}
\end{equation}
%
Isolate $\vec{V}^k$ in the latter and substitute in Eq.\ (\ref{eqn:massconservation}), and simplify noting that $s_k=1$ and $C_k$ is positive for the positive ions: 
%
\begin{equation}
  \frac{\partial N_k}{\partial t} + \sum_{i=1}^3 \frac{\partial}{\partial x_i}  \left(N_k \Delta \vec{V}^k_i+N_k \vec{V}^{\rm n}_i+ s_k N_k \mu_k  \vec{E}_i
             -   \frac{\mu_k}{|C_k|} \frac{\partial P_k}{\partial x_i}\right) = W_k
  \label{eqn:positive_ion_1}
\end{equation}
%
The source terms that must be added to ensure that Gauss's law is satisfied can be obtained by multiplying Gauss's law Eq.\ (\ref{eqn:gauss}) by $\beta_k^{\rm g} \mu_{k} N_{k}$ and rearranging:
%
\begin{equation}
0=\beta_k^{\rm g} \mu_{k} N_{k} \sum_{i=1}^3 \frac{\partial }{\partial x_i}\epsilon_r \vec{E}_i - \beta_k^{\rm g}  \mu_{k} N_{k}\frac{1}{\epsilon_0} \sum_{r=1}^\ns C_r N_r 
\end{equation}
%
or
%
\begin{equation}
0=
 \beta_k^{\rm g} \mu_{k} N_{k}  \sum_{i=1}^3 \frac{\partial \vec{E}_i}{\partial x_i}  
+\beta_k^{\rm g} \mu_{k} N_{k}\frac{1}{\epsilon_r} \sum_{i=1}^3 \vec{E}_i \frac{\partial \epsilon_r}{\partial x_i}  
-\beta_k^{\rm g} \mu_{k} N_{k}\frac{1}{\epsilon_0 \epsilon_r} \sum_{r=1}^\ns C_r N_r 
\end{equation}
%

Then, we add the latter to Eq.\ (\ref{eqn:positive_ion_1}) to obtain:
%
\begin{align}
  \frac{\partial N_k}{\partial t} &+ \sum_{i=1}^3 \frac{\partial}{\partial x_i}  \left(N_k \Delta \vec{V}^k_i+N_k \vec{V}^{\rm n}_i+ s_k N_k \mu_k  \vec{E}_i
             -   \frac{\mu_k}{|C_k|} \frac{\partial P_k}{\partial x_i}\right) \nonumber \\
&= W_k
+\beta_k^{\rm g} \mu_{k} N_{k}  \sum_{i=1}^3 \frac{\partial \vec{E}_i}{\partial x_i}  
-\beta_k^{\rm g} \mu_{k} N_{k}\frac{1}{\epsilon_0 \epsilon_r} \sum_{r=1}^\ns C_r N_r 
+\beta_k^{\rm g} \mu_{k} N_{k}\frac{1}{\epsilon_r} \sum_{i=1}^3 \vec{E}_i \frac{\partial \epsilon_r}{\partial x_i}  
\end{align}
%



And we note that the following statement holds:
%
\begin{equation}
\beta_k^{\rm g} \mu_{k} N_{k} \sum_{i=1}^3 \frac{\partial \vec{E}_i}{\partial x_i}
= \sum_{i=1}^3 \frac{\partial }{\partial x_i} \beta_k^{\rm g} \mu_{k} N_{k} \vec{E}_i
 - \sum_{i=1}^3 \beta_k^{\rm g} \vec{E}_i \frac{\partial }{\partial x_i} \mu_{k} N_{k}
\end{equation}
%
Substitute the latter in the former, and rearrange:
%
\begin{align}
  \frac{\partial N_k}{\partial t} &+ \sum_{i=1}^3 \frac{\partial}{\partial x_i}  \left(N_k \left( \Delta \vec{V}^k_i + \vec{V}^{\rm n}_i+ (s_k-\beta_k^{\rm g}) \mu_k  \vec{E}_i \right)
             -   \frac{\mu_k}{|C_k|} \frac{\partial P_k}{\partial x_i}\right) 
+ \sum_{i=1}^3 \beta_k^{\rm g} \vec{E}_i \frac{\partial }{\partial x_i} \mu_{k} N_{k}
\nonumber \\
&= W_k
-\beta_k^{\rm g} \mu_{k} N_{k}\frac{1}{\epsilon_0 \epsilon_r} \sum_{r=1}^\ns C_r N_r 
+\beta_k^{\rm g} \mu_{k} N_{k}\frac{1}{\epsilon_r} \sum_{i=1}^3 \vec{E}_i \frac{\partial \epsilon_r}{\partial x_i}  
\end{align}
%
The latter is  obtained without introducing assumptions or simplifications from the physical model outlined in Section 2.




\section{Ambipolar Form}

It can be shown that a system of equations composed of the recast transport equation outlined in Eq.\ (\ref{eqn:positivespecies}) for the positive ions, combined with the standard transport equation for the negative species (Eq.\ (\ref{eqn:massconservation})), and combined with a potential equation based on the generalized Ohm's law \cite{jcp:2011:parent} has the same exact solution as the conventional governing equations outlined in Section 3 while not exhibiting high stiffness. However, such a set of equations would yield a rather low resolution and require significantly more nodes to reach the same accuracy within quasi-neutral regions of the plasma. As was  demonstrated in Ref.\ \cite{jcp:2013:parent}, such can be overcome by rewriting the transport equation for the negative species in \emph{ambipolar form} following the approach outlined in Ref.\ \cite{jcp:2011:parent:2}. Rewriting the transport equations in ambipolar form increases the resolution because it reduces the dependence of the potential equation on the charged species transport equations in quasi-neutral regions.     

Without loss of generality, we can add and subtract the ambipolar electric field $\vec{E}^\prime$ to the electric field:
%
\begin{align}
  \frac{\partial N_k}{\partial t} &+ \sum_{i=1}^3 \frac{\partial}{\partial x_i}  \left(N_k \left( \Delta \vec{V}^k_i + \vec{V}^{\rm n}_i+ (s_k-\beta_k^{\rm g}) \mu_k  \vec{E}_i + \beta^{\rm a}_k \mu_k \vec{E}_i^\prime \right)\right) 
\nonumber \\
&+ \sum_{i=1}^3 \beta_k^{\rm g} \vec{E}_i \frac{\partial }{\partial x_i} \mu_{k} N_{k}
- \sum_{i=1}^3  \frac{\partial}{\partial x_i}  \beta_k^{\rm a}\mu_k N_k \vec{E}^\prime_i
- \sum_{i=1}^3  \frac{\partial}{\partial x_i}  \left( 
   \frac{\mu_k }{|C_k|} \frac{\partial P_k}{\partial x_i}
\right)
\nonumber \\
&= W_k
-\beta_k^{\rm g} \mu_{k} N_{k}\frac{1}{\epsilon_0 \epsilon_r} \sum_{r=1}^\ns C_r N_r 
+\beta_k^{\rm g} \mu_{k} N_{k}\frac{1}{\epsilon_r} \sum_{i=1}^3 \vec{E}_i \frac{\partial \epsilon_r}{\partial x_i}  
\label{eqn:negspecies1}
\end{align}
%

As outlined in Ref.\ \cite{jcp:2014:parent}, significant gains in resolution can be reached when $\vec{E}^\prime$ is defined as the component of the electric field that cancels out all components of the current except due to drift. However, defining the ambipolar electric field  in this manner leads to some difficulties when the plasma is in the presence of a magnetic field. Not only does this result in a particularly complicated transport equation in which several terms are expensive to compute, but this  also entails convergence hangs when the magnetic field reaches high values. We here find it necessary to define the ambipolar electric field in a slightly different manner as the component of the electric field that cancels out all components of the \emph{unmagnetized} current except due to drift and due to the motion of the neutrals, with the unmagnetized current being the current that would be obtained locally should the magnetic field be zero. The ambipolar electric field thus takes on the form:
%
\begin{equation}
 \vec{E}_i^\prime \equiv \vec{E}_i - \frac{1}{\sigma} \left(\vec{J}_i-\sum_{r=1}^\ns C_r N_r \Delta \vec{V}_i^r -\rho_{\rm c} \vec{V}_i^{\rm n} \right)
 \label{eqn:Eprimedefinition}
\end{equation}
%
where the term within the bracket on the RHS can be easily shown to be the unmagnetized current density (the current density in the absence of a magnetic field):
%
\begin{equation}
\mfd  \vec{J}_i-\sum_{r=1}^\ns C_r N_r \Delta \vec{V}_i^r  - \rho_{\rm c} \vec{V}_i^{\rm n} =   \sigma \vec{E}_i 
             -  \sum_{r=1}^\ns s_r \mu_r  \frac{\partial P_r}{\partial x_i}
\label{eqn:Jzero}
\end{equation}
%
and where the conductivity $\sigma$ is defined as:
%
\begin{equation}
  \sigma \equiv \sum_{k=1}^\ns \mu_k |C_k| N_k 
\end{equation}
%
After substituting Eq.\ (\ref{eqn:Jzero}) in Eq.\ (\ref{eqn:Eprimedefinition}) it can be shown that:
%
\begin{equation}
   \vec{E}_i^\prime =   
   \sum_{r=1}^\ns \frac{s_r \mu_r }{\sigma}  \frac{\partial P_r}{\partial x_i}
\label{eqn:Eprime}
\end{equation}
%
Then, after substituting $\vec{E}^\prime$ from Eq.\ (\ref{eqn:Eprimedefinition}) and $\vec{E}^\prime$ from Eq.\ (\ref{eqn:Eprime}) into Eq.\ (\ref{eqn:negspecies1}) and rearranging, we obtain:
%
\begin{align}
  \frac{\partial N_k}{\partial t} &+ \sum_{i=1}^3 \frac{\partial}{\partial x_i}  \left(N_k \left( \Delta \vec{V}^k_i + \vec{V}^{\rm n}_i+ (s_k-\beta_k^{\rm g}+ \beta^{\rm a}_k) \mu_k  \vec{E}_i -  \frac{\beta^{\rm a}_k \mu_k}{\sigma} \left(\vec{J}_i-\sum_{r=1}^\ns C_r N_r \Delta \vec{V}_i^r -\rho_{\rm c} \vec{V}_i^{\rm n} \right)  \right)\right) 
\nonumber \\
&+ \sum_{i=1}^3 \beta_k^{\rm g} \vec{E}_i \frac{\partial }{\partial x_i} \mu_{k} N_{k}
- \sum_{i=1}^3  \frac{\partial}{\partial x_i}  \beta_k^{\rm a}\mu_k N_k \sum_{r=1}^\ns \frac{s_r \mu_r }{\sigma}  \frac{\partial P_r}{\partial x_i}
- \sum_{i=1}^3  \frac{\partial}{\partial x_i}  \left( 
   \frac{\mu_k }{|C_k|} \frac{\partial P_k}{\partial x_i}
\right)
\nonumber \\
&= W_k
-\beta_k^{\rm g} \mu_{k} N_{k}\frac{1}{\epsilon_0 \epsilon_r} \sum_{r=1}^\ns C_r N_r 
+\beta_k^{\rm g} \mu_{k} N_{k}\frac{1}{\epsilon_r} \sum_{i=1}^3 \vec{E}_i \frac{\partial \epsilon_r}{\partial x_i}  
\end{align}
%
Rearrange:
%
\begin{align}
  \frac{\partial N_k}{\partial t} &+ \sum_{i=1}^3 \frac{\partial}{\partial x_i}  N_k \left(  \sum_{r=1}^\ns \frac{\delta_{rk}\sigma+ \beta^{\rm a}_k C_r N_r \mu_k}{\sigma}   \Delta \vec{V}_i^r + \left(1+\frac{\beta^{\rm a}_k \mu_k \rho_{\rm c}}{\sigma}\right)\vec{V}^{\rm n}_i + (s_k-\beta_k^{\rm g}+ \beta^{\rm a}_k) \mu_k  \vec{E}_i -  \frac{\beta^{\rm a}_k \mu_k}{\sigma} \vec{J}_i   \right) 
\nonumber \\
&+ \sum_{i=1}^3 \beta_k^{\rm g} \vec{E}_i \frac{\partial }{\partial x_i} \mu_{k} N_{k}
- \sum_{i=1}^3 \sum_{r=1}^{\ns} \frac{\partial}{\partial x_i} \left( \mu_r \left( \frac{\delta_{rk}}{|C_k|}+ \frac{ \beta_k^{\rm a} \mu_k N_k s_r  }{\sigma} \right) \frac{\partial P_r}{\partial x_i}\right) 
\nonumber \\
&= W_k
-\beta_k^{\rm g} \mu_{k} N_{k}\frac{1}{\epsilon_0 \epsilon_r} \sum_{r=1}^\ns C_r N_r 
+\beta_k^{\rm g} \mu_{k} N_{k}\frac{1}{\epsilon_r} \sum_{i=1}^3 \vec{E}_i \frac{\partial \epsilon_r}{\partial x_i}  
\end{align}
%
where $\delta_{rk}$ is the Kronecker delta. 
After splitting the derivative involving the current $\vec{J}$ into two terms and noting that the divergence of the current can be written as:
%
\begin{equation}
\sum_{i=1}^3 \frac{\partial \vec{J}_i}{\partial x_i}=-\sum_{r=1}^\ns C_r \frac{\partial N_r}{\partial t}
\end{equation}
%
we obtain:
%
\begin{align}
  &\sum_{r=1}^{\ns} \frac{\delta_{rk}\sigma+C_r  \beta_k^{\rm a} \mu_k N_k}{\sigma} \frac{\partial N_r}{\partial t}  
  + \sum_{i=1}^3 \frac{\partial}{\partial x_i}  N_k \left(  \sum_{r=1}^\ns \frac{\delta_{rk}\sigma+ \beta^{\rm a}_k C_r N_r \mu_k}{\sigma}   \Delta \vec{V}_i^r + \left(1+\frac{\beta^{\rm a}_k \mu_k \rho_{\rm c}}{\sigma}\right)\vec{V}^{\rm n}_i + (s_k-\beta_k^{\rm g}+ \beta^{\rm a}_k) \mu_k  \vec{E}_i    \right) 
\nonumber \\
&- \sum_{i=1}^3 \beta_k^{\rm a}\vec{J}_i \frac{\partial}{\partial x_i}  \left( \frac{\mu_k N_k}{\sigma} 
\right) 
+ \sum_{i=1}^3 \beta_k^{\rm g} \vec{E}_i \frac{\partial }{\partial x_i} \mu_{k} N_{k}
- \sum_{i=1}^3 \sum_{r=1}^{\ns} \frac{\partial}{\partial x_i} \left( \mu_r \left( \frac{\delta_{rk}}{|C_k|}+ \frac{ \beta_k^{\rm a} \mu_k N_k s_r  }{\sigma} \right) \frac{\partial P_r}{\partial x_i}\right) 
\nonumber \\
&= W_k
-\beta_k^{\rm g} \mu_{k} N_{k}\frac{1}{\epsilon_0 \epsilon_r} \sum_{r=1}^\ns C_r N_r 
+\beta_k^{\rm g} \mu_{k} N_{k}\frac{1}{\epsilon_r} \sum_{i=1}^3 \vec{E}_i \frac{\partial \epsilon_r}{\partial x_i}  
\end{align}
%
We then use the ideal gas relationship in Eq.\ (\ref{eqn:Pk}), expand the partial derivatives, and rewrite:
%
\begin{align}
  &\sum_{r=1}^{\ns} \frac{\delta_{rk}\sigma+C_r  \beta_k^{\rm a} \mu_k N_k}{\sigma} \frac{\partial N_r}{\partial t}  \nonumber\alb
  &+ \sum_{i=1}^3 \frac{\partial}{\partial x_i}  N_k \left(  \sum_{r=1}^\ns \frac{\delta_{rk}\sigma+ \beta^{\rm a}_k C_r N_r \mu_k}{\sigma}   \Delta \vec{V}_i^r + \left(1+\frac{\beta^{\rm a}_k \mu_k \rho_{\rm c}}{\sigma}\right)\vec{V}^{\rm n}_i + (s_k-\beta_k^{\rm g}+ \beta^{\rm a}_k) \mu_k  \vec{E}_i    \right) 
\nonumber \\
&- \sum_{i=1}^3 \beta_k^{\rm a}\vec{J}_i \frac{\partial}{\partial x_i}  \left( \frac{\mu_k N_k}{\sigma} 
\right) 
+ \sum_{i=1}^3 \beta_k^{\rm g} \vec{E}_i \frac{\partial }{\partial x_i} \mu_{k} N_{k}
- \sum_{i=1}^3 \sum_{r=1}^{\ns} \frac{\partial}{\partial x_i} \left(\mu_r k_{\rm B} T_r\left(\frac{ \delta_{rk} }{ |C_k|}+   \frac{ \beta_k^{\rm a}\mu_k N_k s_r}{\sigma}\right)  \frac{\partial N_r}{\partial x_i}\right) 
\nonumber \\
&= W_k
-\beta_k^{\rm g} \mu_{k} N_{k}\frac{1}{\epsilon_0 \epsilon_r} \sum_{r=1}^\ns C_r N_r 
+\beta_k^{\rm g} \mu_{k} N_{k}\frac{1}{\epsilon_r} \sum_{i=1}^3 \vec{E}_i \frac{\partial \epsilon_r}{\partial x_i}  
+ \sum_{r=1}^\ns \sum_{i=1}^3  \frac{\partial}{\partial x_i} \left( \mu_r k_{\rm B} N_r   \left(\frac{\delta_{rk}}{|C_k|}+\frac{s_r \mu_k N_k \beta_k^{\rm a}}{\sigma}\right)  \frac{\partial T_r}{\partial x_i} \right)
\end{align}
%
Multiply by the molecular mass $m_k$ and note that $\rho_k=m_k N_k$:
%
\begin{align}
  &\sum_{r=1}^{\ns} \frac{m_k}{m_r}\frac{\delta_{rk}\sigma+C_r  \beta_k^{\rm a} \mu_k N_k}{\sigma} \frac{\partial \rho_r}{\partial t}  \nonumber\alb
  &+ \sum_{i=1}^3 \frac{\partial}{\partial x_i}  \rho_k \left(  \sum_{r=1}^\ns \frac{\delta_{rk}\sigma+ \beta^{\rm a}_k C_r N_r \mu_k}{\sigma}   \Delta \vec{V}_i^r + \left(1+\frac{\beta^{\rm a}_k \mu_k \rho_{\rm c}}{\sigma}\right)\vec{V}^{\rm n}_i + (s_k-\beta_k^{\rm g}+ \beta^{\rm a}_k) \mu_k  \vec{E}_i    \right) 
\nonumber \\
&- \sum_{i=1}^3 \beta_k^{\rm a}\vec{J}_i \frac{\partial}{\partial x_i}  \left( \frac{\mu_k \rho_k}{\sigma} 
\right) 
+ \sum_{i=1}^3 \beta_k^{\rm g} \vec{E}_i \frac{\partial }{\partial x_i} \mu_{k} \rho_{k}
- \sum_{i=1}^3 \sum_{r=1}^{\ns} \frac{\partial}{\partial x_i} \left(\mu_r k_{\rm B} T_r\frac{m_k}{m_r}\left(\frac{ \delta_{rk} }{ |C_k|}+   \frac{ \beta_k^{\rm a}\mu_k N_k s_r}{\sigma}\right)  \frac{\partial \rho_r}{\partial x_i}\right) 
\nonumber \\
&= m_k W_k
-\beta_k^{\rm g} \mu_{k} \rho_{k}\frac{\rho_{\rm c}}{\epsilon_0 \epsilon_r}  
+\beta_k^{\rm g} \mu_{k} \rho_{k}\frac{1}{\epsilon_r} \sum_{i=1}^3 \vec{E}_i \frac{\partial \epsilon_r}{\partial x_i}  
+ \sum_{r=1}^\ns \sum_{i=1}^3  \frac{\partial}{\partial x_i} \left( \mu_r k_{\rm B} \rho_r  \frac{m_k}{m_r} \left(\frac{\delta_{rk}}{|C_k|}+\frac{s_r \mu_k N_k \beta_k^{\rm a}}{\sigma}\right)  \frac{\partial T_r}{\partial x_i} \right)
\end{align}
%


We can simplify further the latter by defining the ambipolar tensor $\alpha$ as:
%
\begin{equation}
\alpha_{kr} \equiv
  \frac{m_k}{m_r}\frac{\delta_{rk}\sigma+ \beta_k^{\rm a} C_r  \mu_k N_k}{\sigma} 
=  |C_r| \frac{m_k}{m_r}\left( \frac{\delta_{rk}}{|C_k|}+  \frac{ \beta_k^{\rm a} s_r \mu_k N_k}{\sigma}\right)
=  \frac{\rho_k}{\rho_r}\frac{\delta_{rk}\sigma+ \beta_k^{\rm a} C_r  \mu_k N_r}{\sigma} 
\end{equation}
%
and by noting that the following equality holds:
%
\begin{equation}
1+\beta_k^{\rm a} \frac{\mu_k\rho_{\rm c}}{\sigma} 
= \sum_{r=1}^\ns \frac{\delta_{rk}\sigma+ \beta_k^{\rm a} C_r  \mu_k N_r}{\sigma} 
 \end{equation}
%
We thus obtain:
%
\begin{align}
  &  \sum_{r=1}^{\ns} \alpha_{kr} \frac{\partial \rho_r}{\partial t}  
+ \sum_{i=1}^3 \sum_{r=1}^\ns  \frac{\partial}{\partial x_i}      \alpha_{kr} \left( \Delta \vec{V}_i^r + \vec{V}^{\rm n}_i \right)\rho_r  
  + \sum_{i=1}^3 \frac{\partial}{\partial x_i}  \rho_k  (s_k-\beta_k^{\rm g}+ \beta^{\rm a}_k) \mu_k  \vec{E}_i   
\nonumber \\
&- \sum_{i=1}^3 \beta_k^{\rm a}\vec{J}_i \frac{\partial}{\partial x_i}  \left( \frac{\mu_k \rho_k}{\sigma} 
\right) 
+ \sum_{i=1}^3 \beta_k^{\rm g} \vec{E}_i \frac{\partial }{\partial x_i} \mu_{k} \rho_{k}
- \sum_{i=1}^3 \sum_{r=1}^{\ns} \frac{\partial}{\partial x_i} \left(\frac{\mu_r k_{\rm B} T_r \alpha_{kr}}{|C_r|}  \frac{\partial \rho_r}{\partial x_i}\right) 
\nonumber \\
&= m_k W_k
-\beta_k^{\rm g} \mu_{k} \rho_{k}\left(
  \frac{\rho_{\rm c}}{\epsilon_0\epsilon_r}
  -\sum_{i=1}^3 \frac{\vec{E}_i}{\epsilon_r} \frac{\partial \epsilon_r}{\partial x_i}  
\right)
+ \sum_{r=1}^\ns \sum_{i=1}^3  \frac{\partial}{\partial x_i} \left( \frac{\mu_r k_{\rm B} \rho_r  \alpha_{kr}}{|C_r|}   \frac{\partial T_r}{\partial x_i} \right)
\end{align}
%

\section{Matrix Form}

The proposed charged species transport equations can also be written in general matrix form as follows:
%
\begin{equation}
  R=Z\frac{\partial U}{\partial t} + \sum_{i=1}^3\frac{\partial}{\partial x_i} D_i U
    + \sum_{i=1}^3 Y_i^1 \frac{\partial}{\partial x_i} H^1 
    + \sum_{i=1}^3 Y_i^2 \frac{\partial}{\partial x_i} H^2 
    - \sum_{i=1}^3  \frac{\partial}{\partial x_i} \left( K \frac{\partial U}{\partial x_i} \right)-S
\label{eqn:Rproposed}
\end{equation}
%
where $R$ is the residual which is driven to zero through the iterative process, and where the other matrices can be shown to correspond to:
%
\begin{align}
 \left[U\right]_k&=\rho_k \alb
  \left[D_i\right]_{k,k}&=\sum_{r=1}^\ns \frac{\delta_{rk}\sigma+\beta_k^{\rm a} C_r  \mu_k N_r}{\sigma} \left(\Delta \vec{V}_i^r +\vec{V}^{\rm n}_i\right)  +(s_k-\beta_k^{\rm g}+ \beta^{\rm a}_k) \mu_k  \vec{E}_i \alb
% \left[A_i\right]_{k,k}&=&\sum_{r=1}^\ns \alpha_{kr}\frac{N_r}{N_k} \left(\Delta \vec{V}_i^r +\vec{V}^{\rm n}_i\right) \alb
 \left[H^1\right]_{k,k}&= \mu_k \rho_k  \alb
 \left[H^2\right]_{k,k}&= \frac{1}{\sigma} \mu_k \rho_k \alb
 \left[Y_i^1\right]_{k,k}&=\beta_k^{\rm g} \vec{E}_i  \alb
 \left[Y_i^2\right]_{k,k}&=- \beta_k^{\rm a} \vec{J}_i \alb
 \left[K\right]_{k,r} &= \frac{\mu_r k_{\rm B} T_r \alpha_{kr}}{|C_r|} \alb
 \left[Z\right]_{k,r} &= \alpha_{kr} \alb
 \left[S\right]_k &= m_k W_k 
-\beta_k^{\rm g} \mu_{k} \rho_{k}\left(
  \frac{\rho_{\rm c}}{\epsilon_0\epsilon_r}
  -\sum_{i=1}^3 \frac{\vec{E}_i}{\epsilon_r} \frac{\partial \epsilon_r}{\partial x_i}  
\right)
+ \sum_{r=1}^\ns \sum_{i=1}^3  \frac{\partial}{\partial x_i} \left( \frac{\mu_r k_{\rm B} \rho_r \alpha_{kr}}{|C_r|}  \frac{\partial T_r}{\partial x_i} \right)
\end{align}
%
Alternately, we can write the matrix $D$ as:
%
\begin{equation}
 \left[D_i\right]_{k,r}=\alpha_{kr}(\Delta \vec{V}_i^r +\vec{V}_i^{\rm n})+\delta_{kr}(s_k-\beta_k^{\rm g}+ \beta^{\rm a}_k) \mu_k  \vec{E}_i 
\end{equation}
%
In the latter, the velocity difference $\Delta \vec{V}$ can be obtained by substituting Eq.\ (\ref{eqn:V}) into Eq.\ (\ref{eqn:deltaV}):
%
\begin{equation}
 \Delta \vec{V}_i^k = 
   \sum_{j=1}^\nd s_k \wtilde{\mu}^k_{ij}  \vec{E}_j^{\rm n}
      + \sum_{j=1}^\nd  \left(\frac{\delta_{ij} \mu_k-\wtilde{\mu}^{k}_{ij}}{|C_k| N_k}\right) \frac{\partial P_k}{\partial x_j}
-  s_k \mu_k  \vec{E}_i
\end{equation}
%
while the current density is obtained from Eq.\ (\ref{eqn:J}) and the electric field is obtained from the potential equation based on Ohm's law which can be derived from the physical model outlined in Section 2 following the approach shown in Ref.\ \cite{jcp:2011:parent}: 
%
\begin{equation}
\!\begin{array}{l}
  \mfd\sum_{i=1}^3 \frac{\partial}{\partial x_i}\left(\sum_{j=1}^3 \wtilde{\sigma}_{ij} \left(-\frac{\partial \phi}{\partial x_j}  + \left(\vec{V}^{\rm n} \times \vec{B}\right)_j\right) 
             - \sum_{j=1}^3 \sum_{k=1}^\ns s_k   \wtilde{\mu}^{k}_{ij}  \frac{\partial P_k}{\partial x_j}
              + \rho_{\rm c} \vec{V}_i^{\rm n}  \right)=-  \frac{\partial \rho_{\rm c}}{\partial t}
\end{array}
\end{equation}
%
from which the electric field can be found using Eq.\ (\ref{eqn:potential}).

It is noted that the set of equations proposed herein is obtained from the same physical model as the conventional set of equations without introducing any additional assumption or simplification. As such, the exact solution obtained from the proposed set of equations is identical to the one obtained from the conventional set not only for quasi-neutral plasmas with significant ambipolar diffusion and drift phenomena but also for non-neutral sheaths, for unsteady plasmas, as well as for plasmas where the displacement current is non-negligible. 

Despite yielding the same exact solution as the conventional equations, the set of equations proposed herein is advantaged  by being considerably less stiff and hence requiring significantly less computing effort to reach convergence. Further, as will be shown below in the Test Cases section, the proposed equations yield a considerably higher resolution within plasma regions that are quasi-neutral.    





	


\section{Boundary Conditions}


When the electron Hall parameter (i.e.\ the product between the electron mobility and the magnitude of the magnetic field) becomes significant due to a strong applied magnetic field, the enforcement of boundary conditions can become problematic. The difficulties arise when imposing a zero current condition perpendicular to dielectric surfaces by setting to zero the component of $\vec{J}$ perpendicular to the surface in Eq.\ (\ref{eqn:J}). When the magnetic field is non-zero, such leads to the potential $\phi$ at the boundary node depending not only on the properties of the nearest inner node but also on the properties of the adjacent boundary nodes. Numerical experiments show that such direct dependence between boundary nodes entails major convergence problems for many plasma flowfields either when using the conventional or the proposed set of governing equations. The convergence difficulties become more severe when the electron Hall parameter approaches or exceeds 0.1 often leading to the solution diverging towards aphysical states or continuously oscillating without reaching a root. 

One way that this problem can be overcome is by setting the magnetic field to zero on all boundary nodes and on all inner nodes adjacent to boundary nodes. In doing so, the plasma is not subject to a magnetic field near the surfaces and the same boundary conditions as used for a plasma in the absence of magnetic field can be specified:
%
\begin{equation}
\frac{\partial }{\partial \eta} N_+ \vec{V}^{+}_\eta = 0
{~~~~~\rm and~~~~~}
N_{-}=0
{~~~~~\rm and~~~~~}
N_{\rm e}=\frac{\gamma}{\mu_{\rm e}} \sum_{k=1}^\ns N_k \mu_k \beta_k^+
{~~~~~\rm for~}
\vec{E}_\eta<0
\end{equation}
%
%
\begin{equation}
N_{+}=0
{~~~~~\rm and~~~~~}
\frac{\partial }{\partial \eta} N_- \vec{V}^{-}_\eta = 0
{~~~~~\rm and~~~~~}
\frac{\partial }{\partial \eta} N_{\rm e} \vec{V}^{\rm e}_\eta= 0
{~~~~~\rm otherwise} 
\end{equation}
%
with $\gamma$ being the secondary emission coefficient and the subscripts/superscripts ``e'', ``$-$'', and ``$+$'' denoting the electron species, the negative ion species, and the positive ion species respectively. In the latter $\eta$ refers to the coordinate perpendicular to the boundary and pointing away from the surface towards the nearest inner node, while  $\vec{E}_\eta$ and $\vec{V}^k_\eta$ refer to the electric field component and $k$th species velocity component in the direction of $\eta$.   


It may be argued that setting the magnetic field to zero on the boundary and near-boundary nodes may entail some errors in the converged solution. However, such errors disappear as the grid is refined because the volume of the unmagnetized regions near the surfaces becomes insignificant compared to the total volume of the plasma.

When applying the boundary conditions, it is found necessary to under-relax the update of the number densities and of the potential on the boundary nodes in order to prevent convergence hangs. For all cases here considered, either when using the conventional or the proposed equations, the relaxation factor is set to 0, 0.8, and 0.5 for the ion, electron, and potential equations respectively. The relaxation factor is such that, when set to 0, the boundary node property does not depend on its previous value when being updated and, when set to 1, the boundary node property remains unaltered. 





  \bibliographystyle{warpdoc}
  \bibliography{all}


\end{document}
