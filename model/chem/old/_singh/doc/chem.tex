\section{Singh Combustion Model}

The calculation of a chemical source term is explained in detail in Anderson. \cite{gen:anderson2}.
A brief summary of that exaplanation follows here.
Each species conservation equation has a source term for each species $k$ given by:

\begin{displaymath}
\dot{W}_k = {\cal M}_k \sum_{l=1}^{N_e} \left( \nu''_{l,k} - \nu'_{l,k} \right)
\left[ k_{f,l} \prod_{m=1}^{N_l} \left[ \chi_m \right]^{\nu'_{l,m}}
- k_{b,l} \prod_{m=1}^{N_l} \left[ \chi_m \right]^{\nu''_{l,m}} \right]
\end{displaymath}

$N_e$ is the number of elementary reactions involving species $k$.
$N_l$ is the number of species involved in the elementary reaction.
${\cal M}_k$ is the molecular weight of species $k$.
$\chi_m$ is the mole fraction of species $m$.
$\nu'_{l,k}$ is the stoichiometric coefficient for the reactants of reaction $l$.
$\nu''_{l,k}$ is the stoichiometric coefficient for the products of reaction $l$.
$k_b$ is the backward reaction rate constant and
$k_f$ is the forward reaction rate constant, given by the
modified Arrhenius equation:
\begin{displaymath}
k_f = AT^n \exp(-E/{\cal R}T)
\end{displaymath}
where the coefficients $A$, $n$ and $E$ are germane to a particular combustion
model. In this study, the Singh hydrogen-air combustion model is used 
\cite{chem:singh} and the coefficients are given in table \ref{model} 
below.

The backward and the forward reaction rate constants are related by the
equilibrium constant, $K_c$
\begin{displaymath}
\frac{k_f}{k_b} = K_c
\end{displaymath}
The equilibrium constant, $K_c$ is given by:
\begin{displaymath}
K_c = ({\cal R} T)^{-\Delta \nu}
       \exp \left( \frac{-\Delta G^0}{{\cal R} T} \right)
\end{displaymath}

where
\begin{displaymath}
\Delta \nu = \sum_{i=1}^{N_l} (\nu''_i - \nu'_i)
\end{displaymath}
The Gibbs free energy for species $i$
is defined as $G_i = H_i - T S_i$, where $H_i$ is the
enthalpy, $T$ is the temperature and $S_i$ is the entropy.
The difference in Gibbs free energy $\Delta G^0$ of products and reactants is
\begin{displaymath}
\Delta G^0 = \sum_{i=1}^{N_l} (\nu''_i - \nu'_i) \left( h^0_i - T s^0_i \right)
\end{displaymath}
where $h^0_i$ is the molar enthalpy (including the heat of formation at a
reference temperature of 298 K)
of the species $i$, and $s^0_i$ is its the molar entropy,
each calculated at a reference pressure (one atmosphere).\\
\\

\begin{table}
\begin{threeparttable}
\caption{Singh Hydrogen-Air Combustion Model with Nitrogen Inert}
\begin{tabular}{|cc|c|c|c|} \hline
\multicolumn{2}{|c|}{Reaction} & A & n & E  \\ \hline \hline
(1) & H$_{2}$ + O$_{2} \rightarrow$ OH + OH & 1.70 $\times$ 10$^{13}$ & 0 & 48 000 \\
(2) & H + O$_{2} \rightarrow$ OH + O & 1.42 $\times$ 10$^{14}$ & 0 & 16 400 \\
(3) & O + H$_{2} \rightarrow$ OH + H & 2.07 $\times$ 10$^{14}$ & 0 & 13 750 \\
(4) & OH + H$_{2} \rightarrow$ H$_{2}$O + H & 3.16 $\times$ 10$^{7}$ & 1.80 & 3 030 \\
(5) & OH + OH $\rightarrow$ H$_{2}$O + O & 5.50 $\times$ 10$^{13}$ & 0 & 7 000 \\
(6) & H + OH + M $\rightarrow$ H$_{2}$O + M & 2.21 $\times$ 10$^{22}$ & -2.00 & 0 \\
(7) & H + H + M $\rightarrow$ H$_{2}$ + M & 6.53 $\times$ 10$^{17}$ & -1.00 & 0 \\
\hline
\end{tabular}
\label{model}
\end{threeparttable}
\end{table}

The units for $A$ are in $[\frac{cm^{3b}}{(mole-s)^b}]$ where $b=1$ for two body reactions and 
$b=2$ for three body reactions. $E$ is in $[\frac{cal}{mole}]$.\\
\\

The symbol $M$ denotes a third-body collision partner, a species acting as a catalyst only.
The concentration of $M$ is simply determined from the equation :\\

\begin{displaymath}
X_M = \sum_{k=1}^{n_s} \eta_k X_k
\end{displaymath}

where $\eta_k$ is the third-body efficiency. $\eta_k$ is unity
for all species and reactions in this case. 

