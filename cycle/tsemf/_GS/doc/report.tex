\documentclass{warpdoc}
\newlength\lengthfigure                  % declare a figure width unit
\setlength\lengthfigure{0.158\textwidth} % make the figure width unit scale with the textwidth
\usepackage{psfrag}         % use it to substitute a string in a eps figure
\usepackage{subfigure}
\usepackage{rotating}
\usepackage{pstricks}
\usepackage[innercaption]{sidecap} % the cute space-saving side captions
\usepackage{scalefnt}
\usepackage{amsmath}
\usepackage{bm}



%%%%%%%%%%%%%=--NEW COMMANDS BEGINS--=%%%%%%%%%%%%%%%%%%%%%%%%%%%%%%%%%%
\newcommand{\alb}{\vspace{0.2cm}\\} % array line break
\newcommand{\mfd}{\displaystyle}
\newcommand{\nd}{{n_{\rm d}}}
\newcommand{\M}{{\bf M}}
\newcommand{\N}{{\bf N}}
\newcommand{\B}{{\bf B}}
\newcommand{\BI}{\wbar{{\bf B}}}
\newcommand{\A}{{\bf A}}
\newcommand{\C}{{\bf C}}
\newcommand{\T}{{\bf T}}
\newcommand{\Dstar}{D^{\!\star}}
\newcommand{\Fstar}{F^{\!\star}}
\newcommand{\Ustar}{U^{\!\star}}
\newcommand{\Sstar}{S^{\!\star}}
\newcommand{\Kstar}{K^{\!\star}}
\newcommand{\Ystar}{Y^{\!\star}}
\newcommand{\co}{,~~}
\newcommand{\band}{{\rm Band}}
\renewcommand{\fontsizetable}{\footnotesize\scalefont{1.0}}
\renewcommand{\fontsizefigure}{\footnotesize}
\renewcommand{\vec}[1]{\bm{#1}}
\setcounter{tocdepth}{3}
\let\citen\cite

%%%%%%%%%%%%%=--NEW COMMANDS BEGINS--=%%%%%%%%%%%%%%%%%%%%%%%%%%%%%%%%%%

\setcounter{tocdepth}{3}

%%%%%%%%%%%%%=--NEW COMMANDS ENDS--=%%%%%%%%%%%%%%%%%%%%%%%%%%%%%%%%%%%%



\author{
  Bernard Parent
}

\email{
  bernparent@gmail.com
}

\department{
  Dept. of Aerospace and Mechanical Engineering	
}

\institution{
  University of Arizona
}

\title{
  Successive Over Relaxation
}

\date{
  April 2018
}

%\setlength\nomenclaturelabelwidth{0.13\hsize}  % optional, default is 0.03\hsize
%\setlength\nomenclaturecolumnsep{0.09\hsize}  % optional, default is 0.06\hsize

\nomenclature{

  \begin{nomenclaturelist}{Roman symbols}
   \item[$a$] speed of sound
  \end{nomenclaturelist}
}


\abstract{
abstract
}

\begin{document}
  \pagestyle{headings}
  \pagenumbering{arabic}
  \setcounter{page}{1}
%%  \maketitle
  \makewarpdoctitle
%  \makeabstract
%  \tableofcontents
%  \makenomenclature
%%  \listoftables
%%  \listoffigures
\sloppy

\section{Description of the Method}

The Successive Over Relaxation (SOR) method was first outlined in Frankel \cite{moc:1950:frankel} and Young \cite{thesis:1950:young}. It can be summarized as follows.
Say that we want to solve for the system of equations
%
\begin{equation}
A X = B
\end{equation}
%
with $A$ a $n\times n$ square matrix, and $X$ and $B$ some vectors with $n$ rows each. We can split $A$ as a sum of a lower-diagonal, a diagonal, and an upper-diagonal matrix as follows:
%
\begin{equation}
A=L+D+U
\end{equation}
% 
Then:
%
\begin{equation}
(L+D+U) X = B
\end{equation}
%
Multiply all terms by the relaxation factor $\alpha$:
%
\begin{equation}
(\alpha L+\alpha D+ \alpha U) X = \alpha B
\end{equation}
%
Add $(1-\alpha)DX$ on both sides:
%
\begin{equation}
(\alpha L+ D+ \alpha U) X = \alpha B + (1-\alpha)DX
\end{equation}
%
Substract $\alpha U X$ from both sides:
%
\begin{equation}
(\alpha L+ D) X = \alpha B + (1-\alpha)DX- \alpha UX
\end{equation}
%
Regroup:
%
\begin{equation}
(\alpha L+ D) X = \alpha B + \left((1-\alpha)D - \alpha U \right)X
\end{equation}
%
Let's evaluate $X$ on the LHS at the iteration count $m+1$ and $X$ on the RHS at the iteration count $m$:
%
\begin{equation}
(\alpha L+ D) X^{m+1} = \alpha B + \left((1-\alpha)D - \alpha U \right)X^m
\end{equation}
%
In the latter, the relaxation factor $\alpha$ must be set to a value greater than 1 and less than 2.
We can express the latter in tensor form:
%
\begin{equation}
\sum_{k} (\alpha L_{rk} + D_{rk}) X_k^{m+1} = \alpha B_r + \sum_k\left((1-\alpha)D_{rk} - \alpha U_{rk} \right)X_k^m
\end{equation}
%
But $D$ is zero except when $r=k$. Thus:
%
\begin{equation}
D_{rr} X_r^{m+1} + \sum_{k} \alpha L_{rk}  X_k^{m+1} = \alpha B_r + (1-\alpha)D_{rr}X_r^m - \sum_k \alpha U_{rk} X_k^m 
\end{equation}
%
But $L$ is zero wherever $k$ is higher or equal to $r$, and $U$ is zero wherever $k$ is less or equal to $r$:
%
\begin{equation}
D_{rr} X_r^{m+1} + \sum_{k=1}^{r-1} \alpha L_{rk}  X_k^{m+1} = \alpha B_r + (1-\alpha)D_{rr}X_r^m - \sum_{k=r+1}^{n} \alpha U_{rk} X_k^m 
\end{equation}
%
When $k<r$, $L=A$. Also, when $k>r$, $U=A$. Also, when $k=r$, $D=A$. Thus:
%
\begin{equation}
A_{rr} X_r^{m+1} + \sum_{k=1}^{r-1} \alpha A_{rk}  X_k^{m+1} = \alpha B_r + (1-\alpha)A_{rr}X_r^m - \sum_{k=r+1}^{n} \alpha A_{rk} X_k^m 
\end{equation}
%
Regroup:
%
\begin{equation}
A_{rr} X_r^{m+1} = A_{rr}X_r^m -  \alpha A_{rr}X_r^m + \alpha\left(B_r - \sum_{k=1}^{r-1}  A_{rk}  X_k^{m+1}    - \sum_{k=r+1}^{n}  A_{rk} X_k^m \right)
\end{equation}
%
Divide by $A_{rr}$:
%
\begin{equation}
 X_r^{m+1} = X_r^m -  \alpha X_r^m + \frac{\alpha}{A_{rr}}\left(B_r - \sum_{k=1}^{r-1}  A_{rk}  X_k^{m+1}    - \sum_{k=r+1}^{n}  A_{rk} X_k^m \right)
\end{equation}
%
or
%
\begin{equation}
 X_r^{m+1} = (1-\alpha) X_r^m  + \frac{\alpha}{A_{rr}}\left(B_r - \sum_{k=1}^{r-1}  A_{rk}  X_k^{m+1}    - \sum_{k=r+1}^{n}  A_{rk} X_k^m \right)
\end{equation}
%
 

  \bibliographystyle{warpdoc}
  \bibliography{all}


\end{document}
















