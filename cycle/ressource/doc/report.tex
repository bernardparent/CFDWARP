\documentclass{warpdoc}
\newlength\lengthfigure                  % declare a figure width unit
\setlength\lengthfigure{0.158\textwidth} % make the figure width unit scale with the textwidth
\usepackage{psfrag}         % use it to substitute a string in a eps figure
\usepackage{subfigure}
\usepackage{rotating}
\usepackage{pstricks}
\usepackage[innercaption]{sidecap} % the cute space-saving side captions
\usepackage{scalefnt}
\usepackage{amsmath}
\usepackage{bm}

%%%%%%%%%%%%%=--NEW COMMANDS BEGINS--=%%%%%%%%%%%%%%%%%%%%%%%%%%%%%%%%%%
\newcommand{\alb}{\vspace{0.2cm}\\} % array line break
\newcommand{\mfd}{\displaystyle}
\newcommand{\nd}{{n_{\rm d}}}
\newcommand{\M}{{\bf M}}
\newcommand{\N}{{\bf N}}
\newcommand{\B}{{\bf B}}
\newcommand{\BI}{\overline{{\bf B}}}
\newcommand{\A}{{\bf A}}
\newcommand{\C}{{\bf C}}
\newcommand{\T}{{\bf T}}
\newcommand{\co}{,~~}
\newcommand{\band}{{\rm Band}}
\renewcommand{\fontsizetable}{\footnotesize\scalefont{1.0}}
\renewcommand{\fontsizefigure}{\footnotesize}
\renewcommand{\vec}[1]{\bm{#1}}
\setcounter{tocdepth}{3}
\let\citen\cite

%%%%%%%%%%%%%=--NEW COMMANDS BEGINS--=%%%%%%%%%%%%%%%%%%%%%%%%%%%%%%%%%%

\setcounter{tocdepth}{3}

%%%%%%%%%%%%%=--NEW COMMANDS ENDS--=%%%%%%%%%%%%%%%%%%%%%%%%%%%%%%%%%%%%



\author{
  Bernard Parent
}

\email{
  bernparent@gmail.com
}

\department{
  Department of Aerospace and Mechanical Engineering	
}

\institution{
  University of Arizona
}

\title{
  Source Terms Discretization
}

\date{
  September 2016
}

%\setlength\nomenclaturelabelwidth{0.13\hsize}  % optional, default is 0.03\hsize
%\setlength\nomenclaturecolumnsep{0.09\hsize}  % optional, default is 0.06\hsize

\nomenclature{

  \begin{nomenclaturelist}{Roman symbols}
   \item[$a$] speed of sound
  \end{nomenclaturelist}
}


\abstract{
abstract
}

\begin{document}
\sloppy
  \pagestyle{headings}
  \pagenumbering{arabic}
  \setcounter{page}{1}
%%  \maketitle
  \makewarpdoctitle
%  \makeabstract
  \tableofcontents
%  \makenomenclature
%%  \listoftables
%%  \listoffigures



\section{Splitting of the Source Terms to Obtain Positivity}

Consider an ordinary differential equation of the form:
%
\begin{equation}
\frac{d U}{d t}  = S 
\end{equation}
%
where $t$ is time, $S$ is the source term vector, and $U$ is the vector of conserved variables. Discretize the latter using a first-order backward operator for the time derivative (a usual strategy when solving chemically-reacting flow). Then:
%
\begin{equation}
\frac{U_{m}-U_{m-1}}{\Delta t} = S_{m} 
\label{eqn:disceq}
\end{equation}
%
where $m$ refers to the time level and $\Delta t$ to the physical time step.   

Now, let's rewrite $S$ as a function of the flux eigenvectors and eigenvalues:
%
\begin{equation}
 S_{m} = L^{-1}_{m} \Lambda_{m}^S L_{m} U_{m}
\label{eqn:S1}
\end{equation}
%
The latter should be understood as a definition of the $S$ eigenvalues, $\Lambda_{m}^S$. Multiply both sides by $L_{m}$:
%
\begin{equation}
 L_{m} S_{m} =  \Lambda_{m}^S L_{m} U_{m}
\end{equation}
%
or,
%
\begin{equation}
 \left[ L_{m} S_{m} \right]_{r} =  \left[ \Lambda_{m}^S\right]_{r,r} \left[L_{m} U_{m}\right]_r
\end{equation}
%
Isolate the eigenvalues:
%
\begin{equation}
\left[ \Lambda_{m}^S\right]_{r,r}=\frac{ \left[ L_{m} S_{m} \right]_{r}}{\left[L_{m} U_{m}\right]_r}  
\label{eqn:Lambda_S}
\end{equation}
%
It can be easily shown that the discretization equation would conform to the rule of the positive coefficients (and hence lead to a positivity-preserving algorithm) only if all the terms within $\Lambda_{m}^S$ are negative \cite{jcp:2012:parent}. Unfortunately, this is seldom the case and this is why the solution can become tainted with negative mass fractions, negative temperature, or negative pressure. For this reason, we will seek a discretization of the source term of the form:
%
\begin{equation}
  S_{m} = L^{-1}_{m} \Lambda_{m}^- L_{m} U_{m} + L^{-1}_{m-1} \Lambda_{m-1}^+ L_{m-1} U_{m-1}
  \label{eqn:source_split}
\end{equation}
%
where $\Lambda^-$ and $\Lambda^+$ are diagonal matrices composed strictly of negative and positive eigenvalues, respectively. 

To see why this would lead to a positivity-preserving algorithm, substitute the latter in Eq.\ (\ref{eqn:disceq}):
%
\begin{equation}
\frac{U_{m}-U_{m-1}}{\Delta t} = L^{-1}_{m} \Lambda_{m}^- L_{m} U_{m} + L^{-1}_{m-1} \Lambda_{m-1}^+ L_{m-1} U_{m-1}
\end{equation}
%
Put the $U_m$ terms on the LHS and the $U_{m-1}$ terms on the RHS:
%
\begin{equation}
\frac{1}{\Delta t} U_{m} - L^{-1}_{m} \Lambda_{m}^- L_{m} U_{m}
=
   L^{-1}_{m-1} \Lambda_{m-1}^+ L_{m-1} U_{m-1} + \frac{1}{\Delta t} U_{m-1}
\end{equation}
%
Rearrange:
%
\begin{equation}
 L^{-1}_{m} \left(\frac{1}{\Delta t} I -  \Lambda_{m}^-  \right) L_{m} U_{m}
=
   L^{-1}_{m-1} \left( \Lambda_{m-1}^+ + \frac{1}{\Delta t} I \right) L_{m-1} U_{m-1} 
\end{equation}
%
Because the  eigenvalues are strictly positive within both terms, the latter adheres to the rule of the positive coefficients and will hence be positivity preserving. 


\section{Source Vector Splitting}

Perhaps the simplest way the source term eigenvalues $\Lambda^+$ and $\Lambda^-$ can be found is through vector splitting. This is accomplished by first defining the source eigenvalues $\Lambda_{m}^S$ and $\Lambda_{m-1}^S$ such that the following two equations hold:
%
\begin{equation}
 S_{m} = L^{-1}_{m} \Lambda_{m}^S L_{m} U_{m}
\end{equation}
%
%
\begin{equation}
 S_{m} = L^{-1}_{m-1} \Lambda_{m-1}^S L_{m-1} U_{m-1}
\end{equation}
%
From the latter it can be easily shown that the source eigenvalues correspond to:
%
\begin{equation}
\left[ \Lambda_{m}^S\right]_{r,r}=\frac{ \left[ L_{m} S_{m} \right]_{r}}{\left[L_{m} U_{m}\right]_r}  
\end{equation}
%
%
\begin{equation}
\left[ \Lambda_{m-1}^S\right]_{r,r}=\frac{ \left[ L_{m-1} S_{m} \right]_{r}}{\left[L_{m-1} U_{m-1}\right]_r}  
\end{equation}
%
From the latter we can obtain the negative and positive eigenvalues as follows:
%
\begin{equation}
\Lambda_m^- = \frac{1}{2} \left( \Lambda_m^S - |\Lambda_m^S| \right)
\end{equation}
%
%
\begin{equation}
\Lambda_{m-1}^+ = \frac{1}{2} \left( \Lambda_{m-1}^S + |\Lambda_{m-1}^S| \right)
\end{equation}
%


\section{Source Difference Splitting}


We now wish to outline an algorithm which would yield the source terms positive and negative eigenvalues, $\Lambda^+$ and $\Lambda^-$ from the source term $S$.

This is accomplished by first substituting the $S$ eigenvalues outlined in Eq.\ (\ref{eqn:Lambda_S}) in Eq.\  (\ref{eqn:S1}). After some reformatting, the following is obtained:
%
\begin{equation}
S_{m}=
\underbrace{ L^{-1}_{m-1} Y^+_{m-1} L_{m-1} U_{m-1} 
+ L^{-1}_{m} Z^-_{m} L_{m} U_{m}}_\textrm{\small positivity-preserving}
+\underbrace{ L^{-1}_{m-1} Y^-_{m-1} L_{m-1} U_{m-1} 
+ L^{-1}_{m} Z^+_{m} L_{m} U_{m}}_\textrm{\small not~necessarily~positivity-preserving}
\label{eqn:S_split}
\end{equation}
%
with the diagonal matrices $Y^{\pm}$ and $Z^{\pm}$ set equal to: 
%
\begin{equation}
 \left[Y_{m-1}^-\right]_{r,r}  = 0
\label{eqn:Yminus1}
\end{equation}
%
%
\begin{equation}
 \left[Z_{m}^-\right]_{r,r}  = \min\left(0,~\left[ \Lambda_{m}^S\right]_{r,r}\right)
\label{eqn:Zminus1}
\end{equation}
%
%
\begin{equation}
 \left[ Y_{m-1}^+\right]_{r,r}  = 0
\label{eqn:Yplus1}
\end{equation}
%
%
\begin{equation}
 \left[Z_{m}^+\right]_{r,r}  = \max\left(0,~\left[ \Lambda_{m}^S\right]_{r,r}\right)
\label{eqn:Zplus1}
\end{equation}
%
Thus far, the souce terms have not been modified and the stencil is still not positivity-preserving. 

One way that the stencil could be made positivity-preserving is simply by dropping the last two terms on the RHS of Eq.\ (\ref{eqn:S_split}). But, by doing so, the source terms would be modified substantially. For this reason, instead of discarding the two terms that are not necessarily positivity-preserving, let us recast them into new terms, some of which being guaranteed to be positivity-preserving. This can be accomplished by first defining the diagonal matrices $Q$ and $R$ such that the following two statements hold:
%
\begin{equation}
L^{-1}_{m} R_{m} L_{m} U_{m} \equiv L^{-1}_{m-1} Y^-_{m-1} L_{m-1} U_{m-1}
\label{eqn:R_definition}
\end{equation}
%
%
\begin{equation}
L^{-1}_{m-1} Q_{m-1} L_{m-1} U_{m-1} \equiv L^{-1}_{m} Z^+_{m} L_{m} U_{m}
\label{eqn:Q_definition}
\end{equation}
%
Then, using the latter two definitions, the split source terms outlined in Eq.\ (\ref{eqn:S_split}) can be rewritten as:
%
\begin{equation}
S_{m}=
+ L^{-1}_{m-1} (Y^+_{m-1}+Q_{m-1}) L_{m-1} U_{m-1} 
+ L^{-1}_{m} (Z^-_{m}+R_{m}) L_{m} U_{m}
\label{eqn:S_split_2}
\end{equation}
%
where the matrices $R$ and $Q$ can  be obtained in terms of the other matrices by multiplying both sides of Eqs.\ (\ref{eqn:R_definition}) and (\ref{eqn:Q_definition}) by $L_{m}$, writing in tensor form, and then isolating $R$ and $Q$:
%
\begin{equation}
\left[R_{m}\right]_{r,r}  = \frac{\left[L_{m} L^{-1}_{m-1} Y^-_{m-1} L_{m-1} U_{m-1}\right]_r}{\left[L_{m} U_{m}\right]_r}
\label{eqn:R_2}
\end{equation}
%
%
\begin{equation}
\left[Q_{m-1}\right]_{r,r} = \frac{\left[L_{m-1} L^{-1}_{m} Z^+_{m} L_{m} U_{m}\right]_r}{\left[L_{m-1} U_{m-1}\right]_r}
\label{eqn:Q_2}
\end{equation}
%
Now, let us split again $S$ as a sum of positivity-preserving terms and not-necessarily-positivity-preserving terms. This can be done by rewriting Eq.\ (\ref{eqn:S_split_2}) as:
%
\begin{align}
S_{m}&=
\underbrace{ 
 L^{-1}_{m-1} (Y^+_{m-1})^{k+1} L_{m-1} U_{m-1}+L^{-1}_{m} (Z^-_{m})^{k+1} L_{m} U_{m}}_\textrm{\small positivity-preserving}\nonumber\alb
&+\underbrace{  L^{-1}_{m-1} (Y^-_{m-1})^{k+1} L_{m-1} U_{m-1} 
+ L^{-1}_{m} (Z^+_{m})^{k+1} L_{m} U_{m}}_\textrm{\small not~necessarily~positivity-preserving} 
\label{eqn:S_split_3}
\end{align}
%
where the superscript $k$ is an iteration counter such that $(\cdot)^{k+1}$ refers to an update of the properties $(\cdot)$. Then, for Eq.\ (\ref{eqn:S_split_3}) to be equal to the split source terms, Eq.\ (\ref{eqn:S_split_2}), the  updated $Y^\pm$ and $Z^\pm$ diagonal matrices must be equal to: 
%
\begin{equation}
\left[ Y^-_{m-1}\right]^{k+1}_{r,r} = \min \left(0,~\left[Y^+_{m-1}\right]^k_{r,r}+\left[Q_{m-1}\right]^k_{r,r}  \right) 
\label{eqn:Yminus2}
\end{equation}
%
%
\begin{equation}
\left[ Z^-_{m}\right]^{k+1}_{r,r} = \min \left(0,~\left[Z^-_{m}\right]^k_{r,r}+\left[R_{m}\right]^k_{r,r}  \right) 
\label{eqn:Zminus2}
\end{equation}
%
%
\begin{equation}
\left[ Y^+_{m-1}\right]^{k+1}_{r,r} = \max \left(0,~\left[Y^+_{m-1}\right]^k_{r,r}+\left[Q_{m-1}\right]^k_{r,r}  \right) 
\label{eqn:Yplus2}
\end{equation}
%
%
\begin{equation}
\left[ Z^+_{m}\right]^{k+1}_{r,r} = \max \left(0,~\left[Z^-_{m}\right]^k_{r,r}+\left[R_{m}\right]^k_{r,r}  \right) 
\label{eqn:Zplus2}
\end{equation}
%
where the notation  $(\cdot)^k$ denotes the property $(\cdot)$ at the previous iteration count. Then, after substituting $R$ and $Q$ from Eq.\ (\ref{eqn:R_2}) and (\ref{eqn:Q_2}) the latter 4 equations become:
%
\begin{equation}
\left[ Y^-_{m-1}\right]^{k+1}_{r,r} = \min \left(0,~\left[Y^+_{m-1}\right]^k_{r,r}+\frac{\left[L_{m-1} L^{-1}_{m} (Z^+_{m})^k L_{m} U_{m}\right]_r}{\left[L_{m-1} U_{m-1}\right]_r}  \right) 
\label{eqn:Yminus3}
\end{equation}
%
%
\begin{equation}
\left[ Z^-_{m}\right]^{k+1}_{r,r} = \min \left(0,~\left[Z^-_{m}\right]^k_{r,r}+\frac{\left[L_{m} L^{-1}_{m-1} (Y^-_{m-1})^k L_{m-1} U_{m-1}\right]_r}{\left[L_{m} U_{m}\right]_r}  \right) 
\label{eqn:Zminus3}
\end{equation}
%
%
\begin{equation}
\left[ Y^+_{m-1}\right]^{k+1}_{r,r} = \max \left(0,~\left[Y^+_{m-1}\right]^k_{r,r}+\frac{\left[L_{m-1} L^{-1}_{m} (Z^+_{m})^k L_{m} U_{m}\right]_r}{\left[L_{m-1} U_{m-1}\right]_r}  \right) 
\label{eqn:Yplus3}
\end{equation}
%
%
\begin{equation}
\left[ Z^+_{m}\right]^{k+1}_{r,r} = \max \left(0,~\left[Z^-_{m}\right]^k_{r,r}+\frac{\left[L_{m} L^{-1}_{m-1} (Y^-_{m-1})^k L_{m-1} U_{m-1}\right]_r}{\left[L_{m} U_{m}\right]_r}  \right) 
\label{eqn:Zplus3}
\end{equation}
%
By performing several iterations $k=1,2,3,..$, the latter set of equations essentially transforms (as much as possible) the not-necessarily-positivity-preserving terms into positivity-preserving terms (see Eq.\ (\ref{eqn:S_split_3})). However, it is noted that when used in conjunction with Eq.\ (\ref{eqn:S_split_3}), the latter expressions for $Y^\pm$ and $Z^\pm$ will yield exactly the original source terms independently of how many times the matrices $Y^\pm$ and $Z^\pm$ are updated. To make the scheme positivity-preserving, rewrite the source terms, Eq.\ (\ref{eqn:S_split_3}), as:
%
\begin{equation}
S_{m}=
 L^{-1}_{m-1} \Lambda^+_{m-1} L_{m-1} U_{m-1} 
+ L^{-1}_{m} \Lambda^-_{m} L_{m} U_{m}
\label{eqn:S_split_3_2}
\end{equation}
%
with the positive eigenvalues $\Lambda^+$ being set to the sum of the positive wave speeds from both the previous time level $m-1$ and the current time level $m$:  
%
\begin{equation}
\Lambda_{m-1}^+ = Y^+_{m-1} + Z^+_{m} 
\label{eqn:lambdapluQ_2}
\end{equation}
%
and with the negative eigenvalues $\Lambda^-$ defined as the sum of the negative wave speeds originating from both the previous time level $m-1$ and the current time level $m$:
%
\begin{equation}
\Lambda_{m}^- = Z^-_{m} + Y^-_{m-1} 
\label{eqn:lambdaminuQ_2}
\end{equation}
%
Compared to the original source terms, the latter source terms formulation does not modify the wave speed:  the wave speed lost by the source term at the previous iteration is gained by the source term at the next iteration and vice-versa. 

In summary, the ``iterative form'' of the positivity-preserving source terms presented in this section consists of (i) initializing the $Y^\pm$ and $Z^\pm$ diagonal matrices using Eqs.\ (\ref{eqn:Yminus1}) to (\ref{eqn:Zplus1}), (ii) updating the $Y^\pm$ and $Z^\pm$ diagonal matrices through an iterative process by using Eqs.\ (\ref{eqn:Yminus3}) to (\ref{eqn:Zplus3}), (iii) determining the positive and negative eigenvalues through Eqs.\ (\ref{eqn:lambdapluQ_2}) and (\ref{eqn:lambdaminuQ_2}) using the latest updates of the $Y^\pm$ and $Z^\pm$ matrices, and (iv) determining the flux at the interface as in Eq.\ (\ref{eqn:S_split_3_2}). It is here recommended to update the $Y^\pm$ and $Z^\pm$ matrices only two times, as further updating the latter matrices seldomly results in a noticeable improvement of the solution while requiring more computing effort. 

%Finally, it is noted that the flux function presented in this section yields exactly the one presented in Section 3.1 when no iterations are performed (that is, when the $Y^\pm$ and $Z^\pm$ matrices are set as in Eqs.\ (\ref{eqn:Yminus1})-(\ref{eqn:Zplus1}) and not subsequently updated).


\section{Local Pseudotime Step}

Recall the discretization equation Eq.\ (\ref{eqn:disceq}):
%
\begin{equation}
\frac{U_{m}-U_{m-1}}{\Delta t} = S_{m} 
\end{equation}
%
Add a pseudotime derivative:
%
\begin{equation}
\frac{U^{n+1}_m-U_m}{\Delta \tau}
+\frac{U_{m}-U_{m-1}}{\Delta t} = S_{m} 
\end{equation}
%
Split the source term in positive and negative vectors following Eq.\ (\ref{eqn:source_split}):
%
\begin{equation}
  S_{m} = L^{-1}_{m} \Lambda_{m}^- L_{m} U_{m} + L^{-1}_{m-1} \Lambda_{m-1}^+ L_{m-1} U_{m-1}
\end{equation}
%
Substituting the latter in the former:
%
\begin{equation}
\frac{U^{n+1}_m-U_m}{\Delta \tau}
+\frac{U_{m}-U_{m-1}}{\Delta t} = L^{-1}_{m} \Lambda_{m}^- L_{m} U_{m} + L^{-1}_{m-1} \Lambda_{m-1}^+ L_{m-1} U_{m-1}
\end{equation}
%
On  the LHS, keep only the $n+1$ term:
%
\begin{equation}
\frac{1}{\Delta \tau} U^{n+1}_m
 = L^{-1}_{m} \Lambda_{m}^- L_{m} U_{m} + L^{-1}_{m-1} \Lambda_{m-1}^+ L_{m-1} U_{m-1}
-\frac{1}{\Delta t} U_{m}+\frac{1}{\Delta t} U_{m-1}
+\frac{1}{\Delta \tau} U_m
\end{equation}
%
Regroup similar terms together:
%
\begin{equation}
\frac{1}{\Delta \tau} U^{n+1}_m
 = L^{-1}_{m} \left( \Lambda_{m}^- + \frac{1}{\Delta \tau}I -\frac{1}{\Delta t}I\right) L_{m} U_{m} 
 + L^{-1}_{m-1} \left( \Lambda_{m-1}^+ + \frac{1}{\Delta t}I \right) L_{m-1} U_{m-1}
\end{equation}
%
 


\bibliographystyle{warpdoc}
\bibliography{all}


\end{document}
















